\chapter{总结心得}
\section{简要开发节点}
\begin{lstlisting}
    5.20:开会确定了游戏内容、结构以及功能,并且在Github创建项目仓库;完成了前端素材的制作和上传;
    5.30:完成了第一版结构的搭建以及前端界面的实现;
    5.31:实现游戏人物的动态移动、子弹的动态移动以及鼠标键盘响应事件的;
    6.1:完成了项目的重构;实现了中弹功能;
    6.2:完成了人物状态改变、生命值功能以及发弹填弹机制;
    6.5:实现了道具位置随机的效果;
    6.10:完成了三种道具的功能实现;
    6.11:完成击中机制的补充、结算界面的跳转以及一些bug的修复;完成项目文档的编写和代码的整理。
\end{lstlisting}
\section{小组人员分工}
王岩:完成了整体所用数据结构的完整定义,进行代码结构的二次重构以及确定,基本运行逻辑的确定,人物移动,子弹击中、碰撞的初始功能的完成,合作完成了道具随机初始化、击中判定等功能。
\par
杨汶锦:完成了人物状态改变、发弹填弹机制、生命值逻辑判定和生命条界面显示的代码实现;参与了人物中弹的实现;完成了项目的打包以及Latex报告格式规范。
\par
吴一凡:完成了代码整体框架的初次编写,实现了图片和音乐的载入,以及所有界面绘制逻辑,跳转逻辑,按钮逻辑等的编写,合作完成了子弹碰撞,道具随机初始化等功能。
\par
蒙思洁:完成了游戏音乐素材的寻找和图片素材的制作,参与了部分界面绘制逻辑、部分跳转逻辑的撰写,修正了游戏结束判定的逻辑,完成实验报告的框架设计和分解后的模块图绘制。
\section{小组成员心得}
\subsection{王岩}
用汇编来从头写游戏还是十分熬人的,虽然有前人的架构和好用的库,但是真的从头到尾确定数据结构、架构、逻辑过程等等,到一个个细节:理解一个个子函数的作用、一些复杂数据结构如汇编里结构体的调用等等还是非常掉头发的,以及对出其不意的debug还是依靠大家通力合作来解决。
\par
考试时间紧迫,没能对所有功能做完整实现,还是有点遗憾。但是整体实验收获了许许多多的编程经验,对于汇编语言的应用也更加熟练,收获也是很多的。
\subsection{杨汶锦}
由于时间原因,汇编上机作业的小组项目和个人项目几乎是并行的。在两者间不断学习汇编的语法知识、指令使用的细节。在本次小组作业中我们清晰地感受到了将实际需求一步步细分、转化逻辑实现并最终转化代码实现的过程,在项目的各个部分间寻求协调并在一次次调试中寻找逻辑不严谨的地方,整体的开发过程虽然还算顺利但还是令人影响深刻的。
\par
最终我们基本实现了构思初期对这个游戏的绝大部分功能,也对于汇编语言、汇编程序有了更加深刻的认识。在debug时对于一条条指令追本溯源地思考、对于自身逻辑地一遍遍验证,最终实现了这个游戏。感谢默契且给力的队友们!
\subsection{吴一凡}
通过此次汇编小组实验,我对于汇编语言程序设计有了更进一步的认知,学习到了如何使用汇编语言完成一个较大规模程序的编写。在实验初期,由于对于汇编语言结构的不熟悉以及对于代码整体框架的不了解,不知道应该如何组织好整个程序,但在查找了相关资料,并且不断尝试后,我们最终逐步搭起了整个程序的框架。同时由于对一些库函数的不了解,所以在代码编写初期,经常需要查找相关资料,学习如何正确的调用函数。
\par
同样,在整个代码编写过程中我们也遇到了许多问题,也在不断对游戏逻辑进行完善。例如图片无法正确显示,循环无法正确调用,子弹行进路径与预期不符,人物属性出现错误等等。但这些问题最终也都在不断改进,小组成员互帮互助下得以解决。完成了此次实验后,我也收获了许多经验,提高了自己汇编代码的编写能力。
\subsection{蒙思洁}
此次汇编小组的实验令我对汇编语言的了解更深一层,三个个人的小实验令我学到了汇编语言基础的语句操作,而本小组实验令我对汇编语言的开发过程有了更深的认知。通过对整体游戏的效果想象,我们能够首先绘制出游戏每个界面需要什么素材、控件,绘制出主题游戏部分每个素材它的作用是什么,设计它的实现方法;在游戏开发初期需要对素材进行寻找和处理,使其能够合适地被使用,在游戏开发中期需要划分逻辑结构,弄明白每个功能的实现需要哪几方面的共同作用;在游戏开发末期需要对游戏进行大量覆盖性的测试和修正,同时要对应初期设计好的目标来完善那些未完成的功能。
\par
在代码编写中,问题是不可避免的。不论是刚开始搭建框架代码时,基础的图片素材无法被调用、小组成员的环境配置版本不一带来的无法运行的bug;还是后期在测试过程中发现,道具没有正常发挥作用、子弹停留在空中不继续移动的问题、人物的生命值降为0以后不能正确弹出game\_over界面等问题。但是只要肯去钻研,在一行一行代码中捋逻辑,在一个一个重复出现的变量名称中判断它的变化情况,在通力合作之下,我们最终总是能解决这些问题。在这次实验中,队友们的积极沟通和高效合作是我们小组的游戏程序能够完成开发的最重要的组成部分!
\section{不足和展望}
整个项目借鉴了《微观战争》游戏的一部分游戏机制以及操作机制,我们在复现其机制的基础上设计了三种道具特效。
\par
由于整体的项目开发时间短,小组成员精力有限,游戏仍存在以下几方面的不足:
\begin{itemize}
    \item 整体项目采用单线程设计,在高并发操作的情况下可能出现卡顿、崩溃的情况;
    \item 游戏整体动效不够平滑,对于动画关键帧的选取不够准确,游戏看起来较为简单廉价;
    \item 游戏开发过程中的协同不够智能高效,未高效使用git等版本控制技术提高协同开发效率;
\end{itemize}
对于这个游戏,我们还有以下可实现的提高和展望:
\begin{itemize}
    \item 采用多线程重构游戏:由于本次游戏涉及到键盘输入、逻辑计算和动画实现,当出现大量计算或高频次并发键盘输入时,游戏可能会出现卡顿和崩溃的可能性;我们可以采用多线程的方式通过设置全局的标志变量来进行多个线程间的通信,从而提高游戏的鲁棒性;
    \item 补充游戏机制:还可以为游戏增加更加多元化的玩法和彩蛋。比如说,可以在正式游戏界面增加菜单功能,在菜单里可以调整游戏的进度和音乐等效果;可以在玩家中弹以后增加一些成就系统。
\end{itemize}
\section{写在最后}
本次小组项目的实现离不开小组成员的付出和钻研、离不开张全新老师课堂上细致认真的讲授、离不开在GitHub、Google贡献汇编开发技巧和问题解决方案的每一位前辈。感谢各位有形或无形的帮助,祝好!